%\documentclass[journal=jacsat]{achemso}

%\documentclass[aps,prb,twocolumn,amsmath,amssymb,superscriptaddress,longbibliography]{revtex4-1}

%\documentclass[preprint,showpacs,preprintnumbers,amsmath,amssymb]{revtex4-1}
%\documentclass[twocolumn,showpacs,preprintnumbers,amsmath,amssymb]{revtex4}
% Some other (several out of many) possibilities
%\documentclass[preprint,aps]{revtex4}
\documentclass[preprint,aps,draft]{revtex4}
%\documentclass[prb,amsmath,amssymb]{revtex4}% Physical Review B

\usepackage{tikz} %for adding axes labels to figures
\usepackage{graphicx}% Include figure files
\usepackage{epstopdf} %convert graphics
\usepackage{dcolumn}% Align table columns on decimal point
\usepackage{bm}% bold math
\usepackage{amsmath}% bold math
\usepackage{color}
\usepackage{booktabs}
%\usepackage[normalem]{ulem}
%\nofiles
\usetikzlibrary{positioning}

%shortcuts for sets of numbers
\newcommand{\Ns}{\mathbb{N}^{*}}
\newcommand{\N}{\mathbb{N}}
\newcommand{\Z}{\mathbb{Z}}
\newcommand{\Zs}{\mathbb{Z}^{*}}
\newcommand{\R}{\mathbb{R}}
\newcommand{\Rs}{\mathbb{R}^{*}}
\newcommand{\C}{\mathbb{C}}
\newcommand{\Cs}{\mathbb{C}^{*}}

\newcommand{\angstrom}{\text{\normalfont\AA}}
\newcommand\tab[1][1cm]{\hspace*{#1}} %tab shortcut

%\bibliographystyle{achemso}
%\bibliographystyle{unsrt}


%[COMPARE RAW NUMBERS; Elat (the bigger, the more ionic), Ect, etc.]


\begin{document}

\title{
%Using absolutely localized molecular orbitals for deeper physical insight into the nature of chemical bonding and linear-scaling calculations for condensed phase [binary] materials 
%Localized electrons (orbitals) in materials modeling
Using DFT to model $\text{TiO}_2$ nanoparticles
%Absolutely localized orbitals for density functional theory modeling of [binary] materials
}

\author{Nicolas Gastellu}
%\author{Yifei Shi}
\author{Rustam Z. Khaliullin}
\email{rustam.khaliullin@mcgill.ca}
\affiliation{Department of Chemistry, McGill University, 801 Sherbrooke St. West, Montreal, QC H3A 0B8, Canada}

\date{\today}

\begin{abstract} 
%Absolutely localized nonorthogonal molecular orbitals are promising for developing low-cost linear scaling reformulation of Kohn-Sham density functional theory.
%In a recent development, we resolved one of the key issues of has been resolved that lead to the development of a series of efficient DFT methods for weakly-interacting molecular systems. Here, we tested the new method on more challenging strongly interacting systems to understand limits of its applicability. 
\end{abstract}

\maketitle
 
%\section{Outline of research}
%
%Accuracy-speed tests for the following candidate systems with varying importance of charge-delocalization effects. 
%%Unless stated otherwise the systems are static (no molecular dynamics).
%
%\begin{itemize}
%\item Alkali halides
%\item Magnesium hydride
%\item Cadmium selenide
%\item Titanium dioxide
%\item Metal: Magnesium
%%\item Silicon
%\item Boron nitride or analogous system
%%\item Graphene
%%\item Solvated protons (sampling)
%%\item Peptide bonds
%%\item Ionic liquids
%\end{itemize}

\section*{Introduction} 

\section*{Simulation details}

\subsection*{Classical MD}

\tab All classical MD simulations described in this work were ran in the $NPT$ ensemble, using the Matsui-Akaogi (MA) potential to model the interactions between pairs of atoms.
The MA potential was originally developed for classical MD simulations of the four main polytypes of crystalline $\text{TiO}_2$ and has been shown by previous studies to be the most adequate force field for predicting the structure and properties of its liquid and amorphous phases.
The MA potential describes the short-range interaction between atoms with a Buckingham potential and their long-range electrostatic interaction using the traditional Coulomb term:
\begin{equation}
V_{ij}(r) = f_{0}\cdot (B_i+B_j)\cdot\text{exp}\big(\frac{A_i + A_j - r}{B_i + B_j}\big) - \frac{C_{i}C_j}{r^6} + \frac{e\,Z_i\,Z_j}{4\pi\epsilon_0 r}\: ,
\end{equation}
where $r$ denotes the distance between the two interacting ions $i$ and $j$, $e$ is the elementary charge, $Z_i$ is the dimensionless ionic charge of ion $i$, $f_0$ is a standard force of 4.184$\,$kJ$\text{mol}^{-1}\angstrom^{-1}$, and $A_i$, $B_i$, and $C_i$ are parameters corresponding to $i$.
The numerical values of the constants listed above are given in table \ref{classpot}.

\begin{table}[]
\centering
\caption{Parameters used for evaluating the MA potentials.}
\label{classpot}
\
\begin{tabular}{ccccc}
\hline
Element & $Z$ ($|e|$) & $A$ ($\angstrom$) & $B$($\angstrom$) & $C$ $(\angstrom^3\text{kJ}^{1/2}\text{mol}^{-1/2})$ \\ \hline
Ti      & +2.196      & 1.1823            & 0.077            & 22.5                                                \\
O       & -1.098      & 1.6339            & 0.117            & 54.0                                                \\ \hline
\end{tabular}
\end{table}

\tab The atomic structures of the various $a-\text{TiO}_2$ nanoparticles described in this work were obtained in multiple steps.
We started by melting rutile ($r-\text{TiO}_2$) nanocrystals of 198, 390, 768, and 1842 atoms respectively whose structures were previously optimized at the BLYP/DZVP-GTH level of theory (using a plane wave energy cutoff of 2100 Ry) using classical MD in the $NPT$ ensemble with $T = 2000\,$K and $P_{\text{ext},0} = 0.0\,$Pa. 
We ran this first round of simulations for 40000 timesteps of $\Delta t = 0.5\,$fs.
The resulting melt was then cooled in three steps; classical MD simulations using the same potentials, ensemble, and number of steps were ran using $T = 1500\,$K,750$\,$K, and 300$\,$K successively, all with $\Delta t = 2.0\,$fs.
This process simulated the annealing of the melted nanocrystals into a glass which we then studied using Kohn-Sham DFT.

\tab Seeing as the MA potential was originally elaborated to describe the describe the structural properties of $r-\text{TiO}_2$, we also generated a set of conformations of a rutile lattice comprised of 72 atoms in the $NPT$ ensemble at $T = 300\,\text{K}$. 
Doing so provided us with energy values which we expect to be well correlated with the one we obtain with DFT methods. 

[MAYBE PART W RDFs]


\subsection{DFT calculations}

\tab Having obtained equilibriated atomic structures for nanoparticles of different sizes, we sampled 100 conformations from the last 40 ps of the last cooling run, at which point all four nanoparticles were in equilibrium with $T = 300\,\text{K}$ thermal bath.
We then ran single-point energy calculation using KS DFT at the PBE/DZVP level of theory, using a plane wave cutoff of 2000$\,$Ry.
We also ran similar DFT calculations the different configurations of the rutile lattice that we mentioned in the previous section.

%\tab Our results are discussed in the next section and plotted in figures 1-5, in which we plot the DFT energies of every configuration of a given nanoparticle vs the classically computed energies of those same configurations.
%We also added a $y = x$ line to help the reader any kind of correlation between both data sets. 

\section*{Results and discussion}
    
\tab Comparing the energies obtained using the classical MD and those calculated using DFT reveals that while the MA potential adequately describes the structural features of $\text{a-TiO}_2$, it does not allow for an accurate evaluation of the system's energy.
Indeed, plotting the energies yielded by both calculation methods reveals that DFT calculation methods are much more sensitive to a change in a given nanoparticle's atomic configuration than classical methods.
The system for which this is the most obvious is the 768 atom nanoparticle, for which all classically evaluated energies lie within $\approx 10^{-4}\,\text{Ha}$ of each other, while the energies obtained usinq quantum mechanical methods vary by $\approx 10^{-1}\,\text{Ha}$.
While this effect is most dramatic for the 768 atom system, every other nanoparticle on which we ran similar calculations exhibit significant clustering of the energies obtained using the MA potential about their mean value, while their DFT energies spread out over a much larger interval.

\section*{Conclusions} 



\textbf{Acknowledgments.} The research was funded by the Natural Sciences and Engineering Research Council of Canada through the Discovery Grant. The authors are grateful to Compute Canada and McGill HPC Centre for computer time.

%\textbf{Supporting Information}
%Calculated radial distribution functions of liquid water, comparison of timing benchmarks for the DZVP and TZV2P basis sets, timing benchmarks for systems containing 32,768 water molecules, timing benchmarks for the Kohn-Sham matrix build. This material is available free of charge via the Internet at http://pubs.acs.org.

%\bibliography{almo-expand}
%\bibliography{methods}



\end{document}
